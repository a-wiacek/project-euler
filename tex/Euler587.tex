\documentclass[a4paper,12pt]{article}
\usepackage{amsmath}
\usepackage{amssymb}
\begin{document}
\setlength\parindent{0pt}
\textbf{Project Euler 587}
\vspace{5ex}

Let's assume that left bottom corner of rectangle has coordinates \((0, 0)\) and left top corner has coordinates \((0, 2)\). Area of L-section is \(1 - \pi / 4\).

Suppose that rectangle has \(n\) circles inside. Then coordinates of right corners are \((2n, 0)\) and \((2n, 2)\). Let's find points of intersection of diagonal and first circle:

\[
\begin{cases}
(x - 1)^2 + (y - 1)^2 = 1 \\
x = ny
\end{cases}
\]
\[(ny - 1)^2 + (y - 1)^2 = 1\]
\[y^2 (n^2 + 1) - y (2n + 2) + 1 = 0\]
\[\Delta = 4n^2 + 8n + 4 - 4n^2 - 4 = 8n\]
\[y = \frac{n + 1 \pm \sqrt{2n}}{n^2 + 1}\]

From geometric definition of problem we can see, that we are interested in smaller solution.

\[y = \frac{n + 1 - \sqrt{2n}}{n^2 + 1}\]

We can approximate area of concave triangle by area of triangle with base 1 and height \(y\). Approximation is better if \(n\) is large.

\end{document}