\documentclass[a4paper,12pt]{article}
\usepackage{amsmath}
\usepackage{amssymb}
\begin{document}
\setlength\parindent{0pt}
\textbf{Project Euler 233}
\vspace{5ex}

Let \(i\)-prime be a prime number equal to \(i\) modulo 4 and let \(i\)-divisor of \(n\) be a number \(n'\) such that \(n' \mid n\) and \(n' \equiv i\) modulo 4.

Here, will use following theorem: The number of ways of expressing \(n\) as sum of two squares is \(4D(n)\), where \(D(n) = d_1(n) - d_3(n)\) and \(d_i(n)\) is the number of \(i\)-divisors of \(n\).
[I. Niven, H. S. Zuckerman and H. L. Montgomery, An Introduction to the Theory of Numbers, Fifth Edition, John Wiley \& Sons, 1991, pp. 166-167]

We are looking for points \((a, b) \in \mathbb{Z}^2\) such that \((a - N/2)^2 + (b - N/2)^2 = N^2/2\). We have two cases:

\textbf{Case 1:} \(N\) is odd.

Multiplying this equation by 4, we have \((2a - N)^2 + (2b - N)^2 = 2N^2\). Suppose that we found integers \(p\) and \(q\) such that \(p^2 + q^2 = 2N^2\). Since we want to say that \(a = \dfrac{p + N}{2}\) and \(b = \dfrac{q + N}{2}\), we have to check if \(p\) and \(q\) are odd. Suppose that it's not true:
\begin{itemize}
\item Only one of them is odd. Then LHS is odd as sum of odd and even number, but RHS is even.
\item Both \(p\) and \(q\) are even. Then LHS is divisible by 4, but RHS not (only by 2).
\end{itemize}

Therefore all solutions are correct. To count them, we use mentioned theorem. Let \(N = \prod_{i = 1}^k p_i^{a_i} \prod_{j = 1}^l q_j^{b_j}\), where \(p_*\) are 1-primes and \(q_*\) are 3-primes. Then: \[2N^2 = 2 \prod_{i = 1}^k p_i^{2a_i} \prod_{j = 1}^l q_j^{2b_j}\]
Now we will prove that \(D(2N^2) = \prod_{i = 1}^k(2a_i + 1)\) in a following way:
\begin{itemize}
\item It is obviuos that \(d_1(2N^2) = d_1(N^2)\) and \(d_3(2N^2) = d_3(N^2)\).
\item  Let \(s, n \in \mathbb{N}\) and let \(q\) be a 3-prime such that \(q \nmid n\). Then we have \(D(n) = D(nq^{2s})\).

Suppose that \(n\) has \(\gamma\) even divisors, \(\mu\) 1-divisors and \(\xi\) 3-divisors. Then \(nq^{2s}\) has:
\begin{itemize}
\item \(\gamma(2s + 1)\) even divisors: we take even divisor of \(n\) and multiply it by some power of \(q\).
\item \(\mu(s + 1) + \xi s\) 1-divisors: we can either take 1-divisor of \(n\) and multiply it by even power of \(q\) or take 3-divisor of \(n\) and multiply it by odd power of \(q\).
\item \(\xi(s + 1) + \mu s\) 3-divisors: we can either take 3-divisor of \(n\) and multiply it by even power of \(q\) or take 1-divisor of \(n\) and multiply it by odd power of \(q\).
\end{itemize}
Therefore \(D(nq^{2s}) = \mu(s + 1) + \xi s - \xi(s + 1) - \mu s = \mu - \xi = D(n)\).

\item For \(L \in \{0, 1, \dots, l\}\) let \(N_L = \prod_{i = 1}^k p_i^{2a_i} \prod_{j = 1}^L q_j^{2b_j}\). \(L_0\) has \(\prod_{i = 1}^k(2a_i + 1)\) divisors and all are equal to 1 modulo 4, so \(D(N_0) = \prod_{i = 1}^k(2a_i + 1)\). By previous point and induction \(D(N_L) = \prod_{i = 1}^k(2a_i + 1)\) for all \(L\). Since \(N_L = N\), we have proven that \(D(2N^2) = \prod_{i = 1}^k(2a_i + 1)\).
\end{itemize}

We are looking for \(N = Q\prod_{i = 1}^k p_i^{a_i}\), where \(p_*\) are 1-primes, \(Q = \prod_{j = 1}^l q_j^{b_j}\), \(q_*\) are 3-primes and \(4\prod_{i = 1}^k(2a_i + 1) = 420 \leftrightarrow \prod_{i = 1}^k(2a_i + 1) = 105 = 3 \cdot 5 \cdot 7\). We have 5 possibilities:
\begin{itemize}
\item \(2a_1 + 1 = 105\): No solutions, since the lowest 1-prime is 5 and \(5^{52} > 10^{11}\).
\item \(2a_1 + 1 = 15, 2a_2 + 1 = 7 \rightarrow a_1 = 7, a_2 = 3\) (\texttt{case1} in code).
\item \(2a_1 + 1 = 21, 2a_2 + 1 = 5 \rightarrow a_1 = 10, a_2 = 2\): \(13^{10} > 10^{11}\), so we have \(p_1 = 5\) (\texttt{case2} in code).
\item \(2a_1 + 1 = 35, 2a_2 + 1 = 3 \rightarrow a_1 = 17, a_2 = 1\): the lowest possible \(N\) is \(5^{17} \cdot 13 > 10^{11}\), so no solutions.
\item \(2a_1 + 1 = 7, 2a_2 + 1 = 5, 2a_3 + 1 = 3 \rightarrow a_1 = 3, a_2 = 2, a_3 = 1\) (\texttt{case3} in code).
\end{itemize}

\textbf{Case 2}: \(N\) is even.

Let \(N = 2n\). It follows from \((x - y)^2 + (x + y)^2 = 2(x^2 + y^2)\) that \(4D(N) = 4D(n)\). Therefore we can just look at odd \(N\) and multiply it by any power of 2 such that \(2^kN \leq 10^{11}\).
\end{document}