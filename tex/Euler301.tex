\documentclass[a4paper,12pt]{article}
\begin{document}
\setlength\parindent{0pt}
\textbf{Project Euler 301}
\vspace{5ex}

\(X = \oplus\) is xor.

Let \(b_1 b_2 \dots b_k\) be binary representation of \(n\).

Then \(b_1 b_2 \dots b_k 0\) is binary representation of \(2n\).

Since always \(n \oplus 2n \leq 3n\), we can check when equality holds.

For this, there must be \(n \ AND \ 2n = 0 \leftrightarrow
b_i \ AND \ b_{i + 1} = 0\). There must also be \(b_1 = 1\), so \(b_2 = 0\).
Let \(X_m\) denote number of correct integers, which have \(m\) digits
in their binary form. We have \(X_1 = X_2 = X_{31} = 1\).

For \(1 \leq m \leq 28\) we are counting 0 and 1-sequences of length \(m\)
with no two 1s in a row. Let \(X_{m + 2} = a_m + b_m\),
where \(a_m\) is number of sequences of length \(m\) with last digit \(0\)
and \(b_m\) is number of sequences of length \(m\) with last digit \(1\).

Base values are \(a_1 = b_1 = 1\).
Recursive formulas are \(b_{m + 1} = a_m\) (we can't put two 1s in a row)
and \(a_{m + 1} = a_m + b_m\).

One can notice that \(a_m = F_{m + 1}\) and \(b_m = F_m\), where \(F_m\)
is Fibonacci sequence with base values \(F_1 = F_2 = 1\). Therefore:

\[\sum_{k = 1}^{28} X_{k + 2} = \sum_{k = 1}^{28} a_k + b_k = \sum_{k = 1}^{28} F_{k + 2} =
\sum_{k = 1}^{30} F_k - 2 = F_{32} - 3\]

Finally:

\[\sum_{k = 1}^{31} X_k = X_1 + X_2 + \sum_{k = 3}^{30} X_k + X_{31} = F_{32}
= 2178309\]


\vspace{5ex}
\textbf{Answer:}
2178309
\end{document}